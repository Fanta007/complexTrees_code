\documentclass[10pt,a4paper]{article}
\usepackage[T1]{fontenc}
\usepackage[utf8]{inputenc}
\usepackage{textcomp}
\DeclareUnicodeCharacter{00D7}{\texttimes}

\usepackage{amsmath}
\usepackage{amsfonts}
\usepackage{amssymb}
\usepackage{enumerate}
\usepackage{graphicx}
\usepackage{caption}

\topmargin 0.0in
\oddsidemargin 0.0in
\textwidth 6.5in
\textheight 9.0in
\headheight 0.0in
\headsep 0.0in

\setlength\parindent{0pt}

\newcommand{\E}{\mathrm{E}}
\newcommand{\Var}{\mathrm{Var}}
\newcommand{\Cov}{\mathrm{Cov}}
\newcommand{\slfrac}[2]{\left.#1\middle/#2\right.}

\newcommand{\real}{\mathbb{R}}

\title{Simple Tree Data Structure}
\author{Adam Duncan}

\begin{document}

\maketitle



\begin{verbatim}
  >> ST{1}

ans =

  struct with fields:

        t: [1×105 double]
    beta0: [3×105 double]
       T0: 105
        K: 28
     beta: {1×28 cell}
        T: [60 29 31 27 19 23 30 27 19 15 27 18 21 18 24 23 26 17 20 19 15 17 15 12 8 8 6 7]
       tk: [1×28 double]
        d: 3

\end{verbatim}

Fields in ST structure:
\begin{itemize}
  \item {\bf \texttt{t}}:
  Vector of length \texttt{T0} giving the current parameterization of the main branch. Typically this is initialized with something like \texttt{linspace(0,1,T0)}.

  \item {\bf \texttt{beta0}}:
  A \texttt{d\texttimes T0} matrix giving the main branch curve. The $i^\text{th}$ column,  \texttt{beta0(:,i)}, gives the coordinates in $\real^d$ of $\beta_0(t)$ at $t$ given by \texttt{t(i)}.

  \item {\bf \texttt{T0}}:
  The number of points in the discretization of the main branch.

  \item {\bf \texttt{K}}:
  The number of side branches.

  \item {\bf \texttt{beta}}:
  A \texttt{1\texttimes K} cell array of matrices representing the \texttt{K} side branch curves. The $k^\text{th}$ matrix, \texttt{beta\{k\}}, is a \texttt{d\texttimes T(k)} matrix giving the points in the discretization of $\beta_k$.

  \item {\bf \texttt{T}}:
  A vector of \texttt{K} integers where \texttt{T(k)} gives the number of points in the discretization of $\beta_k$

  \item {\bf \texttt{tk}}:
  A vector of length \texttt{K}, giving the starting points of the side branches in terms of their parameter values along the main branch. That is, if \texttt{tk(k)} is $t_k$, then $\beta_0(t_k)=\beta_k(0)$.

  \item {\bf \texttt{d}}:
  The number of spatial dimensions in which the tree lies, typically equal to 3.
\end{itemize}

\end{document}
